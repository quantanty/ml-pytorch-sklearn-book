\subsection{Key Takeaways}
\begin{itemize}
    \item CNNs are highly effective for vision-ralated problems.
    \item Convolution layer, pooling layer.
    \item Regularization for CNN: L1, L2, dropout.
    \item Data augmentation: flipping, rotating, scaling, and cropping.
    \item Building CNN classes in Pytorch with \texttt{nn.Sequential} or \texttt{nn.Module}
    \item Save and load models
\end{itemize}

\subsection{Challenges Encountered}

Despite experimenting with various models, the highest accuracy achieved was not significantly greater than the initial accuracy.

From this chapter onwards, the network architectures will be very complex, and the data will be very large. This creates a training time problem.

This is the first time writing a chapter report, so my writing skills are still lacking, and it takes time to learn each part.

\subsection{Opening Questions}
\begin{itemize}
    \item What strategies can be used to identify the optimal architecture and hyperparameters for a specific problem?
    \item What techniques can be applied to further enhance model accuracy?
    \item What are the best practices for managing and cleaning dirty data in real-world scenarios?
\end{itemize}


